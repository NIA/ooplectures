\documentclass[a4paper, 14pt, titlepage]{extarticle}
  \usepackage{cmap}
  \usepackage[hidelinks,pdftex,unicode]{hyperref}
  \usepackage{mathtext} % для кириллицы в формулах
  \usepackage[T2A]{fontenc}
  \usepackage[utf8]{inputenc}
  \usepackage[english,russian]{babel}
  \usepackage{indentfirst}
  \usepackage{cite}
  \usepackage{amsmath} % для \eqref
  \usepackage{amssymb} % для \leqslant
  \usepackage{amsthm} % для \pushQED
  \usepackage{color} % пока только для TODO:
  \usepackage[pdftex]{graphicx}
  \usepackage{subfig}
  \usepackage{numprint}
  \usepackage[left=3cm,right=2cm,top=2cm,bottom=2cm,bindingoffset=0cm]{geometry}
  \usepackage{datetime}
  \usepackage[normalem]{ulem} % красивые подчёркивания, но обычный курсив в \em, \emph
  \usepackage{listings}
    \lstloadlanguages{Java, Ruby}
    \definecolor{dkgreen}{rgb}{0,0.6,0}
    \definecolor{dkblue}{rgb}{0,0,0.6}
    \definecolor{maroon}{rgb}{0.5,0,0}
    \definecolor{ltgray}{rgb}{0.8,0.8,0.8}
    \lstset{extendedchars=true,    % включаем не латиницу
            escapechar=|,          % между этими символами - код \LaTeX
            frame=tb,              % рамка сверху и снизу
            basicstyle=\footnotesize, % основной шрифт
            commentstyle=\itshape\color{dkgreen}, % шрифт для комментариев
            stringstyle=\color{maroon}, % шрифт для строк
            keywordstyle=\bfseries\color{blue}, % шрифт для ключевых слов
            breaklines=true,       % включить авто переносы
            breakatwhitespace=false, % разрешить переносить не только по пробелу
            showstringspaces=false % не подчёркивать пробелы в строках
            }

  \graphicspath{{../img/}{../../img/}}
  \frenchspacing

  \DeclareSymbolFont{T2Aletters}{T2A}{cmr}{m}{it} % кириллица в формулах курсивом

  \addto\captionsrussian{
    \renewcommand\contentsname{Содержание}
    \renewcommand\lstlistingname{Листинг}
    % перекрываю \refname, чтобы список литературы сам добавлял себя в оглавление
    \let\oldrefname\refname
    \renewcommand\refname{\addcontentsline{toc}{section}{\oldrefname}\oldrefname}
  }

  \newcommand{\underscore}[1]{\hbox to#1{\hrulefill}}
  \newcommand{\todo}[1]{\textbf{\textcolor{red}{TODO: #1}}}
  \newcommand{\note}[1]{\textit{Примечание: #1}}
  \newcommand{\eng}[1]{{\English #1}}

  % обёртка с моими настройками поверх figure:
  % \begin{myfigure}{подпись}{fig:label} ... \end{myfigure}
  \newenvironment{myfigure}[2]%
    {\pushQED{\caption{#1} \label{#2}} % push caption & label
     \begin{figure}[!htb]\centering } %
    {  \popQED % pop caption & label
     \end{figure}}

  % вставка картинки: \figure[params]{подпись}{file}
  % создаёт label вида fig:file
  \newcommand{\includefigure}[3][]{
    \begin{myfigure}{#2}{fig:#3}
      \includegraphics[#1]{#3}
    \end{myfigure}
  }

  % вставка subfigure внутри myfigure:
  % \subfigure[params]{подпись}{file}
  \newcommand{\subfigure}[3][]{
    \subfloat[#2]{\label{fig:#3}\includegraphics[#1]{#3}}
  }

  \newcommand{\vect}[1]{\vec{#1}} % единое выделение векторов (стрелкой)
  \newcommand{\matx}[1]{\mathbf{#1}} % единое выделение матриц (полужирным)
  \newcommand{\transposed}{\top} % единый знак транспонирования (U+22A4 down tack)
  \renewcommand{\le}{\leqslant} % <= с наклонной нижней перекладиной
  \renewcommand{\ge}{\geqslant} % >= с наклонной нижней перекладиной

  \linespread{1.3}

  % русские буквы для списков и частей рисунка
  \renewcommand{\theenumii}{(\asbuk{enumii})}
  \renewcommand{\labelenumii}{\asbuk{enumii})}
  \renewcommand{\thesubfigure}{\asbuk{subfigure}}

  \setcounter{tocdepth}{3} % глубина оглавления

  \bibliographystyle{gost780u}

  \hyphenation{англ} % убрать перенос в этом сокращении

  % алиас и настройки для numprint
  \newcommand{\num}[1]{\numprint{#1}}
  \npthousandsep{\,}
  \npthousandthpartsep{}
  \npdecimalsign{,}

  \newcommand{\checkdate}[3]{({\Russian дата обращения: \formatdate{#1}{#2}{#3}})}

  % выделение ещё более важных моментов, чем \emph
  \newcommand{\strong}[1]{\textbf{#1}}
  % подчёркивание определяемого в определении
  \newcommand{\define}[1]{\uwave{#1}}

  % общий способ написания "то есть": сокращённый или нет
  \newcommand{\ie}{т.~е.~}

  % участок с увеличенными полями
  \newenvironment{indented}%
    { \begingroup %
        \noindent %
        \leftskip2em %
        \rightskip\leftskip }%
    { \par\endgroup }

  % выделение примеров
  \newenvironment{example}%
    { \begin{indented} %
        \color{dkgreen} %
        \small %
        \textbf{\underline{Пример:}} }%
    { \end{indented} }

  % выделение дополнительной, "продвинутой" информации, необязательной при первом прочтении
  \newenvironment{extrainfo}%
    { \begin{indented} %
        \color{dkblue} %
        \small %
        \textbf{\textcircled{\footnotesize i}} }%
    { \end{indented} }

  % пол-страничная по ширине область (для листингов)
  \newenvironment{halfpage}%
    {\noindent\begin{minipage}[h]{0.49\linewidth}} %
    {\end{minipage}\hfill}

  % код прямо в строке: \inlinecode[Java]{int abc;}
  \newcommand{\inlinecode}[2][Java]{\lstinline[basicstyle=\ttfamily, language=#1]{#2}}

  % список без отступа слева
  \newenvironment{compactitemize} %
    {\begin{list}{\labelitemi}{\leftmargin=1em}}%
    {\end{list}}
  

  \author{И.\,А.\,Новиков}
  \title{Лекция <<Классы в Java и Ruby>>}

\begin{document}

%----------------------- титульный лист ------------------------

  \begin{titlepage}
  \begin{center}
    Цикл лекций <<ООП в Java и Ruby>>\\
    \url{https://github.com/NIA/ooplectures}

    \vspace{4cm}
    
    \includegraphics[height=5cm]{ruby-in-cup}

    Иван Новиков

    {\Large Классы в Java и Ruby}

    Лекция №4

    {\small Версия 0.2 от \today}

  \end{center}

    \begin{flushright}

    \end{flushright}

    \vspace{3cm}
    
    \begin{indented}
      \small Для лучшего понимания содержимого лекции знание основ синтаксиса языков Java и Ruby,
      на уровне главы 3 книги \eng{Core Java (C.\,Horstmann, G.\,Cornell)} и главы 2 части I книги
      \eng{Programming Ruby: The Pragmatic Programmers' Guide (D. Thomas)} \emph{или} прочтение
      предыдущих лекций.
    \end{indented}

    \vfill
  
  \begin{center}

    Краснодар~--- 2012~год

  \end{center}
  \end{titlepage}

%----------------------- лекция ------------------------
  \section{Зачем П~--- ОО?}

  Или, говоря по-русски, зачем нужно объектно-ориентированное программирование (ООП)?

  Затем, что таково человеческое мышление:
  \begin{itemize}
    \item Компьютеру проще работать с \strong{числами}, адресами, инструкциями, в лучшем случае~--- с процедурами.
    \item Человеку же привычнее выделять в окружающей реальности \strong{объекты}, путём абстракции
    выделять их свойства и способы взаимодействия с~ними, а~также выделять общее и
    \strong{классифицировать}.
  \end{itemize}

  Именно поэтому языки низкого уровня (\ie наиболее близкие к машинному представлению кода) обычно % TODO Pascal?
  процедурные: С, Pascal, Fortran; а языки высокого уровня (ЯВУ) чаще объектно-ориентированные:
  \footnote{хотя есть ещё функциональные и логические! Не стоит про них забывать.}
  C++, C\#, Java, Python, Ruby...

  \section{Основные понятия. Принцип Абстракции}\label{sec:basic_concepts}

  Итак, человек выделяет объекты, \ie отделяет тот или иной объект от остальной реальности (и от
  других объектов, в частности). Общее определение объекта дать сложно, для программирования же
  важно, что объект~--- это сущность, чётко отделяемая как от других сущностей, так и от себе
  подобных, и имеющая вполне определённые \strong{свойства} и \strong{способы поведения}.

  Важно то, что человек никогда не пытается при осознании объекта выделить \emph{все} его свойства и
  методы поведения (это и невозможно), а сосредотачивается только на тех, которые интересуют его в
  данный момент, отбрасывая остальные.

  \begin{example}
    Держа в руках яблоко и собираясь его съесть, человек думает о его вкусе, цвете и, возможно,
    массе и калорийности, но его вовсе не интересует его электрический заряд, спектральный состав,
    а~также тот факт, что его можно выкинуть в окно, а не съесть.
  \end{example}

  Это и лежит в основе \define{принципа абстракции}: при описании объектов в ООП довольствуются
  минимальным необходимым набором свойств и методов поведения, отбрасывая остальные.

  Среди рассматриваемых объектов можно выделить такие, у которых набор свойств и методов одинаков, а
  отличаются только значения свойств. Объекты такие совершенно одинаковы с точки зрения взаимодействия % непрозрачно
  с ними. Множество таких объектов называют \define{классом}.

  Важно отметить, что так как набор свойств и методов у объектов одного класса одинаков, то можно,
  не меняя смысла, говорить как о наборе свойств/методов \emph{объекта}, так и о наборе
  свойств/методов его \emph{класса}.

  \begin{example}
    \emph{Вася}~--- объект класса \emph{Человек}. У него есть такие свойства, как \emph{имя} и
    \emph{возраст}. Логично, что \emph{имя} и \emph{возраст} есть также и у \emph{Пети}, и у
    \emph{Насти}, и вообще у всех объектов класса \emph{Человек}. Поэтому можно сказать, что
    \emph{имя, возраст}~--- это не просто свойства \emph{Васи, Пети и Насти}~--- это свойства
    \emph{Человека}, \ie класса.
  \end{example}

  Поскольку не бывает объектов, не принадлежащих ни к какому классу, то вообще не принято говорить о
  свойствах и методах \emph{объекта}~--- вместо этого говорят о свойствах и методах \emph{класса},
  \ie о свойствах и методах, которыми обладают все без исключения объекты этого класса.

  Итак, для того, чтобы определить класс, необходимо задать его имя, список его свойств, методов и
  способ создания новых объектов (точнее~--- их инициализации, \ie установки всех свойств в
  начальные значения и произведения некоторых стартовых действий, применения настроек). Этот способ
  сам является особым методом класса и называется \define{конструктором}.

  \begin{extrainfo}
    В некоторых языках (C++, Objective C), где управление памятью происходит явно (явно выделяется,
    явно освобождается), наравне с конструктором у классов есть \define{деструктор}, отвечающий за
    освобождение памяти и всех ресурсов, занятых в конструкторе.
  \end{extrainfo}

  Важной особенностью некоторых свойств и методов является то, что они отвечают внутреннему
  состоянию объекта, и не должны изменяться извне. Их называют \define{приватными} (\eng{private}),
  в~отличие от \define{публичных} (\eng{public}), \ie доступных всем.

  \begin{example}
    У класса \emph{Тетрадь} есть свойства \emph{владелец} и \emph{формат}. Мы легко можем передать
    тетрадь другому владельцу, переписав значение соответствующего свойства. Но формат тетради не
    может \strong{сам} стать больше от того, что мы присвоим в свойство \emph{формат} новое
    значение, так что такое присваивание приведёт к тому, что объект будет иметь некорректное
    состояние. Поэтому свойство \emph{формат} лучше сделать приватным.
  \end{example}

  Объединение данных и кода, который их обрабатывает, а также сокрытие их и защита от
  несанкционированного использования лежат в основе \define{принципа инкапсуляции}.

  \section{Классы в Java и Ruby}

  Определение класса в этих языках происходит сходным образом: сначала определяется имя, а затем
  открывается блок кода, так что всё находящееся в нём относится к классу:

  \begin{halfpage}
    \begin{lstlisting}[language=Java, title={Класс в Java}, gobble=6, texcl]
      // Блок кода~--- это \{ ... \}
      
      class Person {
        // ...
      }
    \end{lstlisting}
  \end{halfpage}
  \begin{halfpage}
    \begin{lstlisting}[language=Ruby, title={Класс в Ruby}, gobble=6, texcl]
      # Блок кода~--- это begin ... end
      
      class Person
        # ...
      end
    \end{lstlisting}
  \end{halfpage}

  Объекты в программировании хранятся в памяти программы, а ссылки на них~--- в переменных и
  константах. Понятно, что класс объекта будет при этом типом переменной, в которой хранится этот объект.

  Итак, объявляя класс, мы объявляем новый тип данных и теперь можем объявлять переменные этого типа:

  \begin{halfpage}
    \begin{lstlisting}[language=Java, title={Переменные в Java}, gobble=6, texcl]
      Person vasya;
      // так же как с обычными типами:
      int age;
      String name;
      // String~--- тоже класс, часть стандартной библиотеки
    \end{lstlisting}
  \end{halfpage}
  \begin{halfpage}
    \vskip 5mm
    В Ruby нет необходимости объявлять
    переменные, так как это динамический язык
  \end{halfpage}

  \begin{extrainfo}
    Важно отметить, что объявленная таким образом переменная ещё не указывает ни на какой объект. Пока
    в неё ничего не было присвоено, её значение остаётся \inlinecode[Java]{null}. Чтобы что-то
    присвоить, как правило, нужно это что-то создать, а это мы научимся делать чуть позже.
  \end{extrainfo}

  \subsection{Объявление методов}

  Вернёмся к объявлению класса. Проще всего обстоит дело с объявлением его методов: оно выглядит
  похоже в Ruby и Java, и представляет собой объявление функции, помещённое в тело класса.

  \begin{halfpage}
    \begin{lstlisting}[language=Java, title={Метод в Java}, gobble=6, texcl]
      class Person {
        public String fullName(boolean f) {
          // ...
        }
      }
    \end{lstlisting}
    \footnotesize
    \begin{compactitemize}
      \item public~--- показывает, что метод публичный
      \item String~--- тип возвращаемого значения
      \item fullName~--- имя метода
      \item boolean~--- тип параметра
      \item f~--- имя параметра
    \end{compactitemize}
  \end{halfpage}
  \begin{halfpage}
    \begin{lstlisting}[language=Ruby, title={Метод в Ruby}, gobble=6, texcl]
      class Person
        def full_name(f)
          # ...
        end
      end
    \end{lstlisting}
    \footnotesize
    \begin{compactitemize}
      \item (по умолчанию публичный)
      \item (тип возвращаемого значения не нужен)
      \item full\_name~--- имя метода
      \item (тип параметра не нужен)
      \item f~--- имя параметра
    \end{compactitemize}
  \end{halfpage}
  \vskip 5mm % TODO без этого хака

  Теперь, если \inlinecode{vasya}~--- переменная, указывающая на объект класса Person, то мы можем
  вызвать этот метод от этого объекта (одинаково в Java и Ruby): \inlinecode{vasya.fullName(true)}.
  Здесь \inlinecode{true}~--- это значение, которое передаётся в метод в качестве параметра.

  Вызвать этот метод без объекта мы не можем. Дело в том, что при вызове метода кроме явных
  параметров (как \inlinecode{f} здесь) в него неявно передаётся ссылка на объект, от которого он
  вызван, так что изнутри метода можно менять значение полей этого объекта или вызывать от него
  другие методы. В Java эта ссылка называется \inlinecode[Java]{this} (по-русски <<этот>>), в
  Ruby~--- \inlinecode[Ruby]{self} (по-русски <<сам>>).

  \begin{halfpage}
    \begin{lstlisting}[language=Java, title={Метод в Java}, gobble=6, texcl]
      class Person {
        public String fullName(boolean f) {
          return this.surname() + ...
        }
      }
    \end{lstlisting}
  \end{halfpage}
  \begin{halfpage}
    \begin{lstlisting}[language=Ruby, title={Метод в Ruby}, gobble=6, texcl]
      class Person
        def full_name(f)
          return self.surname() + ...
        end
      end
    \end{lstlisting}
  \end{halfpage}

  Если вызвать этот метод, как выше: \inlinecode{vasya.fullName(true)},
  то \inlinecode{f} будет равно \inlinecode{true}, а \inlinecode[Java]{this} (\inlinecode[Ruby]{self})
  будет указывать на тот же объект, что и \inlinecode{vasya}, как будто мы сделали
  \inlinecode[Java]{this = vasya;} (\inlinecode[Ruby]{self = vasya}).

  \subsection{Объявление свойств}

  Свойство представляется переменной, объявленной в контексте класса. Если в классе
  \inlinecode{Person} объявлено свойство \inlinecode{name}, то оно, по определению, есть у любого
  объекта класса Person, так что в него можно присвоить значение (\inlinecode{vasya.name = "Vasya"})
  или, наоборот, считать его (\inlinecode{s = vasya.name}), \ie оно ведёт себя как переменная,
  привязанная к данному конкретному объекту. У другого объекта её значение может быть другим, и
  т.~д. Ещё такое поведение делает свойство похожим на \emph{поле} в таблице свойств, поэтому свойство
  также называют \define{полем} (field). Ещё его иногда называют переменной экземпляра
  (\define{\eng{instance variable}}).

  Поскольку переменные в Java объявляются заранее, а в Ruby~--- только при использовании, то это
  относится и к свойствам, из-за чего синтаксис работы с ними в Java и Ruby сильно отличается.

  \begin{halfpage}
    \begin{lstlisting}[language=Java, title={Свойство в Java}, gobble=6, texcl]
      class Person {
        // объявление свойства name
        public String name = "";
        
        public void setName(String value) {
          // чтение внутри класса
          String old = this.name;
          // присваивание внутри класса
          this.name = value;
        }
      }
      ...
      // использование вне класса
      vasya.name = "Vasya";
    \end{lstlisting}
  \end{halfpage}
  \begin{halfpage}
    \begin{lstlisting}[language=Ruby, title={Свойство в Ruby}, gobble=6, texcl]
      class Person
        # объявление свойства name
        # (не нужно)

        def set_name(value)
          # чтение внутри класса
          old = @name
          # присваивание внутри класса
          @name = value
        end
      end
      ...
      # использование вне класса: напрямую нельзя!
      # обходной путь~--- instance\_variable\_get/set
    \end{lstlisting}
  \end{halfpage}

  Как видно, поля в Java, как и методы, могут быть private, и тогда к им нельзя доступаться извне
  класса. В Ruby же они все по умолчанию приватные. Как же вообще работать с приватными свойствами?
  Через методы, поскольку изнутри методов они доступны полностью.

  Среди различных методов, работающих с приватными полями, выделяют два: тот, который просто
  возвращает его значение и тот, который принимает новое значение и присваивает его в
  переменную-поле. Первый в Java принято называть getter'ом, поскольку его имя традиционно имеет вид
  get\_поле, а второй~--- setter'ом, по аналогичной причине. В Ruby принято называть первый
  reader, а второй~--- writer, а оба они~--- accessor'ы. Ридер обычно называется так же, как поле, а
  райтер~--- как ридер, с добавлением знака <<=>>.

  Объявление обоих аксессоров для приватного поля в соответствии с принятыми традициями будет
  выглядеть так:

  \begin{halfpage}
    \begin{lstlisting}[language=Java, title={Геттеры/сеттеры в Java}, gobble=6, texcl]
      class Person {
        private String name = "";
        
        public String getName() {
          return this.name;
        }

        public void setName(String value) {
          this.name = value;
        }

        // --- ИЛИ ---


        // В Java, увы, нет таких удобных сокра-
        // щений, как в Ruby, заменяющих объявле-
        // ние геттеров и сеттеров.

        // Зато их можно легко сгенерировать из
        // контекстного меню большинства сред
        // разработки для конретного свойства.

      }
    \end{lstlisting}
  \end{halfpage}
  \begin{halfpage}
    \begin{lstlisting}[language=Ruby, title={Аксессоры в Ruby}, gobble=6, texcl]
      class Person
        # (объявлять не надо)

        def name()
          return @name
        end
        
        def name=(value)
          @name = value
        end

        # --- ИЛИ ---

        # эти два вызова сгенерируют
        # такие же два метода, как выше
        attr_reader :name
        attr_writer :name

        # --- ИЛИ ---

        # сгенерировать оба сразу
        attr_accessor :name
      end
    \end{lstlisting}
  \end{halfpage}

  Таким образом будет выглядеть вызов этих методов:

  \begin{halfpage}
    \begin{lstlisting}[language=Java, title={Вызов геттера/сеттера в Java}, gobble=6, texcl]
      String s = vasya.getName();
      vasya.setName(s);
    \end{lstlisting}
  \end{halfpage}
  \begin{halfpage}
    \begin{lstlisting}[language=Ruby, title={Вызов аксессоров в Ruby}, gobble=6, texcl]
      s = vasya.name
      vasya.name = s
    \end{lstlisting}
  \end{halfpage}

  В примере для Ruby мы воспользовались тем, что в Ruby можно опускать скобки при вызове функции
  (благодаря чему вместо \inlinecode{vasya.name()} стоит просто \inlinecode{vasya.name}), а также
  тем, что использование оператора присваивания таким образом автоматически преобразуется к вызову
  метода \inlinecode{name=}, как если бы мы написали \inlinecode{vasya.name=(s)}. Всё это было
  сделано для того, чтобы работа с аксессорами выглядела компактнее и имела такой вид, как будто это
  просто публичное свойство. Потому публичных свойств и нет в Ruby: \inlinecode{attr_accessor :name}
  даёт полностью аналогичное поведение.

  \begin{extrainfo}
    У читателя может возникнуть вопрос: а зачем \strong{в Java} может понадобиться объявлять одновременно и
    геттер, и сеттер, причём тривиальные, как выше, если полный доступ к свойству можно открыть проще,
    сделав его \inlinecode{public}, а не \inlinecode{private}?

    При кажущейся на первый взгляд эквивалентности, не стоит забывать, что код программы меняется и
    развивается вместе с программой по мере её жизненного цикла. В \strong{перспективе}
    использование геттеров и сеттеров вместо оставления поля публичным имеет следующие преимущества:

    \begin{enumerate}
      \item Проще отлаживать~--- при возникновении ошибки можно легко отследить каждое обращение к
            полю при помощи точки останова (breakpoint) или отладочной печати.
      \item Проще править~--- можно легко найти в коде все случаи использования каждого из этих методов.
      \item Проще менять доступ~--- если захочется запретить полный доступ, достаточно убрать сеттер
            или сделать его приватным. Не придётся заменять все обращения к полю на геттер, он уже и
            так используется
      \item Проще развивать~--- если внутреннее устройство геттера/сеттера усложнится (например,
            если усложнится внутреннее устройство данных или добавятся проверки), опять же, не
            придётся менять простое обращение на геттер в зависящем коде.
    \end{enumerate}
  \end{extrainfo}

  \subsection{Статические поля и методы}

  Отдельный вид полей и методов~--- это \define{статические} поля и методы. Они, в отличие от
  обычных, не связаны ни с каким объектом, а связаны с классом. Если обычное поле называют
  переменной экземпляра (\eng{instance variable}), то статическое~--- переменной класса (\eng{class
  variable}). При их использовании вместо переменной-объекта используется имя класса. Таким образом,
  такое поле имеет как бы общее значение для всех объектов класса, так же и результат вызова
  статического метода никак не зависит от состояния объектов этого класса, их может даже вообще ещё
  не быть. Статическое поле~--- это, по сути, просто глобальная переменная, а статический метод~---
  глобальная функция, только они объявлены в контексте класса.

  \begin{halfpage}
    \begin{lstlisting}[language=Java, title={Объявление static'ов в Java}, gobble=6, texcl]
      class Person {
        public final static String BIO_NAME
            = "Homo Sapiens"; // (1)
        public static long population
            = 6000000000; // (2)
        public static long malePopulation()
        { // (3)
          ...
        }
      }
    \end{lstlisting}
  \end{halfpage}
  \begin{halfpage}
    \begin{lstlisting}[language=Ruby, title={Объявление static'ов в Ruby}, gobble=6, texcl]
      class Person
        BIO_NAME = "Homo Sapiens" # (1)
        # НЕТ (2)
        def Person.malePopulation() # (3)
        end
        # ИЛИ
        def self.malePopulation() # (3) \phantomsection \label{lst:static_self}
        end

      end
    \end{lstlisting}
  \end{halfpage}

  Здесь (1)~--- это объявление статической константы, \ie неизменяемого статического поля; (2)~---
  объявление переменной класса, \ie изменяемого статического поля (в Ruby оно не требуется, как и
  для обычной переменной); (3)~--- объявление статического метода.

  \begin{halfpage}
    \begin{lstlisting}[language=Java, title={Использование static'ов в Java}, gobble=6, texcl]
      String s = Person.BIO_NAME; // (4)
      n = Person.malePopulation(); // (5)
      long n = Person.population; // (6)
      class Person {
        public static malePopulation() {
          return population/2; // (7)
        }
      }
    \end{lstlisting}
  \end{halfpage}
  \begin{halfpage}
    \begin{lstlisting}[language=Ruby, title={Использование static'ов в Ruby}, gobble=6, texcl]
      s = Person::BIO_NAME # (4)
      n = person.malePopulation() # (5)
      # НЕТ (6)
      class Person
        def Person.male_population()
          return @@population/2 # (7)
        end
      end
    \end{lstlisting}
  \end{halfpage}
  
  Здесь (4)~--- это использование статической константы (обратите внимание, что в Java используется
  точка, как и для обычных полей и методов, а в Ruby~--- \inlinecode[Ruby]{::}), (5)~--- вызов
  статического метода, (6)~--- использование статического поля вне класса, (7)~--- использование
  статического поля изнутри класса (в Java при этом можно опускать имя класса и вместо
  \inlinecode{Person.population} писать просто \inlinecode{population}). Здесь, так же как и в
  случае с обычными полями, имеется значительное отличие Ruby от Java: напрямую статические поля
  недоступны извне, для работы с~ними нужно использовать аксессоры. Их объявление будет выглядеть
  аналогично тому, как это было для обычных полей, только и поле, и метод будут статическими:

  \begin{center}
  \begin{halfpage}
    \begin{lstlisting}[language=Ruby, title={Статические аксессоры в Ruby}, gobble=6, texcl]
      class Person
        def Person.population()
          return @@population
        end
        def Person.population=(value)
          @@population = value
        end
        # инициализация (начальное значение)
        @@population = 6000000000
      end

      # После этого можно писать
      n = Person.population
      Person.population = n
    \end{lstlisting}
  \end{halfpage}
  \end{center}

  Так же, как и в случае с обычными полями, последние две строчки выглядят, как прямой доступ к полю,
  но на самом деле являются сокращённой записью вызова соответствующих методов, объявленных выше.
  К~сожалению, в Ruby нет аналогов \inlinecode{attr_reader}, \inlinecode{attr_writer} и
  \inlinecode{attr_accessor} для генерации аксессоров к статическим полям. Однако, в
  библиотеке Active Support, являющейся частью фреймворка Ruby on Rails, таковые имеются. Называются
  они так же, как вышеперечисленные, только с добавлением буквы \inlinecode{c} (от class) в начале: 
  \inlinecode{cattr_reader} и т.п. Так что при использовании Rails (или отдельно Active Support) ими
  можно пользоваться.

  \begin{extrainfo}
    Неожиданной особенностью статических полей в Ruby является то, что их значение является общим
    не только для всех объектов данного класса, но и для всех объектов классов-наследников этого
    класса. Иными словами, присваивая статическое поле с таким же именем в классе-наследнике, мы
    тем самым изменяем и значение этого поля в классе-родителе (подробнее об этом можно почитать 
    \href{http://apidock.com/rails/Class/cattr_accessor#624-Important-note}{\uline{здесь}}, на англ.
    языке, там же описывается, как можно использовать вместо статических полей другую технику, чтобы
    избежать проблемы). Что это значит, станет понятнее после прочтения следующей лекции о наследовании.
  \end{extrainfo}

  Поскольку статический метод, в отличие от обычного, вызывается не от объекта, а от класса, то в
  него не передаётся ссылка на объект в виде \inlinecode[Java]{this}/\inlinecode[Ruby]{self}.
  Поэтому в Java ключевое слово \inlinecode[Java]{this} вообще недоступно в теле статических методов,
  а в Ruby \inlinecode[Ruby]{self} указывает на объект, описывающий текущий класс, \ie в данном
  примере на \inlinecode{Person} (литералы классов в Ruby являются константными объектами системного
  класса \inlinecode{Class}). Именно поэтому при объявлении статического метода перед его именем
  можно вместо имени класса писать \inlinecode[Ruby]{self}, как это было показано выше \hyperref[lst:static_self]{(3)}.

  В любом случае, из статического метода нельзя обращаться ни к каким полям и методам, не являющимся
  также статическими.

  \subsection{Создание новых объектов}

  Как и говорилось раньше (см. \ref{sec:basic_concepts}), для определения класса, кроме уже
  рассмотренных объявления имени класса, свойств и методов, нужно ещё описать способ создания новых
  объектов. Это реализуется путём объявления специального метода, который самым первым будет вызван
  от объекта, как только он создан~--- \define{конструктора}. Его объявление в обоих языках выглядит
  как объявление метода со специальным именем. Имя это в Java и Ruby разное: в Java оно совпадает с
  именем класса, а в Ruby оно фиксированное: \inlinecode[Ruby]{initialize}. В Java ещё одним
  отличием конструктора от обычного метода является отсутствие возвращаемого типа.

  \begin{halfpage}
    \begin{lstlisting}[language=Java, title={Трививальный конструктор в Java}, gobble=6, texcl]
      class Person {
        public Person() {

        }
      }
    \end{lstlisting}
  \end{halfpage}
  \begin{halfpage}
    \begin{lstlisting}[language=Ruby, title={Тривиальный конструктор в Ruby}, gobble=6, texcl]
      class Person
        def initialize()

        end
      end
    \end{lstlisting}
  \end{halfpage}

  Выше был объявлен конструктор, который не принимает никаких аргументов и ничего не делает. Такой
  простой конструктор можно даже не объявлять, он будет автоматически добавлен компилятором или
  интерпретатором, если не будет объявлен другой, и называется конструктором по умолчанию.

  На самом деле, конструктор обычно используется, чтобы инициализировать объект, \ie привести его в
  рабочее начальное состояние: установить начальные значения свойств и произвести какие-то действия,
  чётко привязанные к моменту его создания (например, зарегистрировать его в каком-то реестре,
  записать в журнал о его создании и т.п.).

  Будучи методом, конструктор может принимать аргументы, что используется обычно для задания
  начальных значений полей. Эти значения передаются в качестве аргументов.

  Рассмотрим пример конструктора для \inlinecode{Person}, который принимает параметр~--- имя
  человека, обрабатывает его и присваивает в соответствующее поле. В качестве примера
  дополнительного действия в конструкторе у нас будет увеличиваться на единицу значение статической
  переменной \inlinecode{population}, чтобы её значение в любой момент было равно количеству
  созданных к этому моменту объектов \inlinecode{Person}.

  \begin{halfpage}
    \begin{lstlisting}[language=Java, title={Конструктор с параметром в Java}, gobble=6, texcl]
      class Person {
        public String name;
        public static int population = 0;
        
        public Person(String name) {
          this.name = name.trim();
          population += 1;
        }
      }
    \end{lstlisting}
  \end{halfpage}
  \begin{halfpage}
    \begin{lstlisting}[language=Ruby, title={Конструктор с параметром в Ruby}, gobble=6, texcl]
      class Person
        attr_reader :name
        @@population = 0

        def initialize(name)
          @name = name.strip
          @@population += 1
        end
      end
    \end{lstlisting}
  \end{halfpage}

  \begin{extrainfo}
    Обратите внимание, что в примере на Java и свойство, и аргумент функции имеют одинаковое имя.
    Это не запрещено, но при этом переменная (аргумент) \inlinecode{name} как бы перекрывает
    свойство \inlinecode{name}, и к свойству \inlinecode{name} уже нельзя обратиться без
    \inlinecode[Java]{this}.
  \end{extrainfo}

  \begin{extrainfo}
    Хотя выше было сказано, что использовать в Java приватные поля с соответствующими геттерами и
    сеттерами в большинстве случаев лучше, чем публичные поля, здесь используются публичные поля для
    краткости.
  \end{extrainfo}

  Теперь можно создать объект класса \inlinecode{Person} с именем Vasya и присвоить его в
  переменную \inlinecode{vasya}. Создание объекта в обоих языках похоже на вызов метода, но выглядит
  по-разному.

  \begin{halfpage}
    \begin{lstlisting}[language=Java, title={Создание объекта в Java}, gobble=6, texcl]
      Person vasya = new Person("Vasya");
    \end{lstlisting}
  \end{halfpage}
  \begin{halfpage}
    \begin{lstlisting}[language=Ruby, title={Создание объекта в Ruby}, gobble=6, texcl]
      vasya = Person.new("Vasya")
    \end{lstlisting}
  \end{halfpage}

  Как видим, в Java для создания объекта используется ключевое слово \inlinecode[Java]{new}, в Ruby
  же \inlinecode[Ruby]{new}~--- это статический метод, который есть у любого класса.

  В обоих случаях эта строчка делает одно и то же: сначала выделяет новый кусочек памяти, где будет
  храниться объект, а затем вызывает его конструктор.
  
  Понятно, что если у конструктора нет параметров (как в первом примере), то всё выглядит так же,
  только без списка аргументов.

  \begin{halfpage}
    \begin{lstlisting}[language=Java, title={Создание объекта в Java}, gobble=6, texcl]
      Person vasya = new Person();
    \end{lstlisting}
  \end{halfpage}
  \begin{halfpage}
    \begin{lstlisting}[language=Ruby, title={Создание объекта в Ruby}, gobble=6, texcl]
      vasya = Person.new # скобки опущены
    \end{lstlisting}
  \end{halfpage}

  \vspace{1cm}

  \strong{Ура!} Всего вышесказанного достаточно, чтобы объявлять классы, создавать их объекты и
  работать с ними, используя их свойства и методы.

  \begin{center}
    \begin{halfpage}
      \hrulefill
    \end{halfpage}
  \end{center}

  Отдельное дополнение стоит сделать насчёт Java. Помимо того, что начальные значения полей можно
  задавать в конструкторе, их также можно задать при определении поля, как если бы эта была
  переменная. Но при этом можно использовать для присваивания только постоянные значения.

  \begin{halfpage}
    \begin{lstlisting}[language=Java, title={Инициализация поля в Java}, gobble=6, texcl]
      class Person {
        public int age = 0;
        
        // это эквивалентно

        public Person {
          this.age = 0;
        }
      }
    \end{lstlisting}
  \end{halfpage}
  \begin{halfpage}
    В Ruby нет аналогичной возможности, поскольку свойства здесь не объявляются заранее.
  \end{halfpage}

  Эта возможность позволяет обойтись вообще без объявления конструктора, если он состоит только из
  подобных присваиваний констант в поля. Все их можно заменить инициализацией полей при объявлении и
  обойтись конструктором по умолчанию.

\end{document}

